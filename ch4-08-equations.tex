\section{$A_{LL}$ Methodology}

We begin with the equation for a double spin asymmetry,
%
\begin{equation}
  A_{LL} = \frac{\sum_{runs} P_{Y}P_{B}\left[(N_{uu} + N_{dd}) - R(N_{ud} + N_{du})\right]}{\sum_{runs} P_{Y}^{2}P_{B}^{2}\left[(N_{uu} + N_{dd}) + R(N_{ud} + N_{du})\right]}.
\end{equation}
%
Henceforth we employ parity conservation; the \(++\) subscript denotes a sum
over the \(uu\) and \(dd\) states and the \(+-\) subscript a sum over \(ud\)
and \(du\) states. \(P_Y\) and \(P_B\) are the Yellow and Blue beam
polarizations, \(N_{ij}\) are spin-dependent charged pion yields, and \(R =
\frac{\mathcal{L}_{++}}{\mathcal{L}_{+-}}\) is a ratio of sampled luminosities
in different spin configurations. Each of the $\mathcal{L}_{ij}$ is a sum of
the scaler counts from board 5 over bunch crossings with that spin state. The
sum is restricted to a set of timebins chosen to approximate a 60~cm cut on
the $z$ position of the vertex; in the 2005 analysis timebins 7, 8, and 9 were
selected, while the 2006 analysis added timebin 6 into the sum.

The formula for the statistical uncertainty on \(A_{LL}\) neglects
uncertainties on the relative luminosities and beam polarizations. Assuming
Poisson statistics on \(N_{++}\) and \(N_{+-}\) the uncertainty on the
asymmetry for a single run is
%
\begin{equation}
  \left(\frac{\sigma_{A_{LL}}}{A_{LL}}\right)^2 = \left(N_{++} + R^2 N_{+-}\right)\left[\frac{1}{N^2} + \frac{1}{D^2} - \frac{2 \times COV(N,D)}{ND} \right]
\end{equation}
%
where \(N\) and \(D\) are the numerator and denominator of the raw asymmetry
(that is, neglecting polarizations), and \(COV(N,D) = N_{++} - R^2 N_{+-}\).
In the case of small asymmetries (\(\frac{1}{N^2} \gg \frac{1}{D^2} - \frac{2
COV(N,D)}{ND}\)), the uncertainty on the numerator dominates the uncertainty
on \(A_{LL}\):
%
\begin{equation}
  \left(\frac{\sigma_{A_{LL}}}{A_{LL}}\right)^2 \approx \frac{N_{++} + R^2 N_{+-}}{N^2}
\end{equation}
%
ROOT's TH1::Divide method does the error propagation correctly in this limit
(it ignores the covariance term). The generalization to a sum over runs is
straightforward since the yields for each run are uncorrelated.

\subsection{Multi-Particle Statistics}

This analysis often accepts multiple pions from a single event. Treating each
of these particles as an independent event and na\"ively using
$\sqrt{N_{pions}}$ for the errors as in the previous section is not correct.
Following the prescription in \cite{sowinski-multiparticle-statistics} we fill
each bin in a histogram at most once per event, using a weight equal to the
number of particles that fell into the bin. This technique neglects
correlations across bins.

\subsection{Background Subtraction}

Next we consider asymmetries with sideband subtraction. Specifically, the raw
charged pion asymmetries in this analysis are contaminated by protons, kaons,
and electrons. The protons and kaons are necessarily combined into a single
sideband. Both sidebands have some non-negligible pion contribution. We start
by defining a reduced background fraction that accounts for the impurities in
the sideband:

\begin{equation*}
  f_{x}(y) = \frac{x~\mathrm{counts~in}~y~\mathrm{window}}{\mathrm{total~ in}~y~\mathrm{window}}
\end{equation*}

\begin{equation*}
  f'(x) = \frac{f_{x}(\pi)}{1 - f_{\pi}(x)}
\end{equation*}
%
The standard equations for the background-subtracted $A_{LL}^{\pi}$ and its
statistical uncertainty are only modified by replacing the background fraction
for each sideband with its reduced background fraction:

\begin{equation}
  A_{LL}^{\pi} = \frac{ A_{LL}^{\pi,raw} - f'(p+K)A_{LL}^{p+K,raw} - f'(e)A_{LL}^{e,raw} }{1 - f'(p+K) - f'(e) }
  \label{eqn:all}
\end{equation}

\begin{equation}
  \sigma_{A_{LL}^{\pi}} = \frac{\sqrt{ \sigma_{A_{LL}^{raw}}^{2} + f'(p+K)^{2} * \sigma_{A_{LL}^{p+K,raw}}^{2} + f'(e)^{2} * \sigma_{A_{LL}^{e,raw}}^{2} }}{1 - f'(p+K) - f'(e)}
  \label{eqn:sigma-all}
\end{equation}
