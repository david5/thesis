% -*-latex-*-
% $Log: cover.tex,v $
% Revision 1.8  2008/05/13 15:02:15  jdreed
% Degree month is June, not May.  Added note about prevdegrees.
% Arthur Smith's title updated
%
% Revision 1.7  2001/02/08 18:53:16  boojum
% changed some \newpages to \cleardoublepages
%
% Revision 1.6  1999/10/21 14:49:31  boojum
% changed comment referring to documentstyle
%
% Revision 1.5  1999/10/21 14:39:04  boojum
% *** empty log message ***
%
% Revision 1.4  1997/04/18  17:54:10  othomas
% added page numbers on abstract and cover, and made 1 abstract
% page the default rather than 2.  (anne hunter tells me this
% is the new institute standard.)
%
% Revision 1.4  1997/04/18  17:54:10  othomas
% added page numbers on abstract and cover, and made 1 abstract
% page the default rather than 2.  (anne hunter tells me this
% is the new institute standard.)
%
% Revision 1.3  93/05/17  17:06:29  starflt
% Added acknowledgements section (suggested by tompalka)
% 
% Revision 1.2  92/04/22  13:13:13  epeisach
% Fixes for 1991 course 6 requirements
% Phrase "and to grant others the right to do so" has been added to 
% permission clause
% Second copy of abstract is not counted as separate pages so numbering works
% out
% 
% Revision 1.1  92/04/22  13:08:20  epeisach

% NOTE:
% These templates make an effort to conform to the MIT Thesis specifications,
% however the specifications can change.  We recommend that you verify the
% layout of your title page with your thesis advisor and/or the MIT 
% Libraries before printing your final copy.

%%%%\title{Objects and Communication Services Modelling for the Automatic Speed 
%%%%Regulator of a typical Itaipu's Hydraulic Turbine implementing the \\IEC
%%%% 61850
%%%%Standard}
%%%%\title{Objects and Communication Services Modelling for the Automatic Speed 
%%%%Regulator of a typical Itaipu dam's Hydraulic Turbine implementing the \\IEC
%%%%61850 Standard}

%\title{Modelado de nodos l�gicos y servicios de comunicaci�n
%del regulador de velocidad de una unidad generadora t�pica de Itaipu
%en conformidad con la norma IEC 61850}

\title{Dise�o del modelo IEC 61850 del regulador de velocidad 
de una unidad generadora t�pica de la 
Central Hidroel�ctrica Itaipu}

%%otro titulo alternativo podria ser
%%dise~no del modelo IEC
%%61850 del regulador de velocidad
%%de una unidad generadora tpica de
%%Itaipu

%Oootro t�tulo alternativo podr�a ser 
%dise�o del modelo de nodos l�gicos y servicios 
%de comunicaci�n del regulador de velocidad 
%de una unidad generadora t�pica de itaipu
%en conformidad con la norma IEC 61850


\author{David Daniel P\'erez Sosa}
% If you wish to list your previous degrees on the cover page, use the 
% previous degrees command:
%       \prevdegrees{A.A., Harvard University (1985)}
% You can use the \\ command to list multiple previous degrees
%       \prevdegrees{B.S., University of California (1978) \\
%                    S.M., Massachusetts Institute of Technology (1981)}
\department{Ingenier�a El�ctrica}

% If the thesis is for two degrees simultaneously, list them both
% separated by \and like this:
% \degree{Doctor of Philosophy \and Master of Science}
\degree{Ingeniero Electricista}

% As of the 2007-08 academic year, valid degree months are September, 
% February, or June.  The default is June.

\degreemonth{Diciembre}
\degreeyear{2010}
\thesisdate{Diciembre, 2010}

%% By default, the thesis will be copyrighted to MIT.  If you need to copyright
%% the thesis to yourself, just specify the `vi' documentclass option.  If for
%% some reason you want to exactly specify the copyright notice text, you can
%% use the \copyrightnoticetext command.  
%\copyrightnoticetext{\copyright IBM, 1990.  Do not open till Xmas.}
\copyrightnotice{David Daniel P\'erez Sosa, 2010}
% If there is more than one supervisor, use the \supervisor command
% once for each.
%\supervisor{Ladislao Aranda Arriola}{Itaipu Binacional}
%\supervisor{Rodrigo Ramos}{Itaipu Binacional}
%\supervisor{Juan Manuel Ramirez Duarte}{Facultad Polit\'ecnica}
%\supervisor{Ladislao Aranda Arriola}{Electrical Engineer - Itaipu Binacional}
%\supervisor{Rodrigo Ramos}{Electrical Engineer - Itaipu Binacional}
%\supervisor{Juan Manuel Ramirez Duarte}{Electrical Engineer - Facultad
% Polit\'ecnica}
\supervisor{Prof. M.Sc. Ing. Rodrigo Ramos}{�rea de ingenier�a, Itaipu
Binacional} 
\supervisor{Prof. Ing. Ladislao Aranda Arriola}{�rea de ingenier�a, Itaipu
Binacional}
\supervisor{Prof. Ing. Juan Manuel Ramirez Duarte}{Profesor Titular, Facultad
Polit\'ecnica}
 
% This is the department committee chairman, not the thesis committee
% chairman.  You should replace this with your Department's Committee
% Chairman.
\chairman{Prof. Dr. Anastasio Sebasti\'an Arce Encina}{Coordinador de
Trabajo Final de Grado de Ingenier�a El�ctrica}

% Make the titlepage based on the above information.  If you need
% something special and can't use the standard form, you can specify
% the exact text of the titlepage yourself.  Put it in a titlepage
% environment and leave blank lines where you want vertical space.
% The spaces will be adjusted to fill the entire page.  The dotted
% lines for the signatures are made with the \signature command.
\maketitle

%%La dedicatoria, de momento no agrego, debido a que no cumple 
%%con el standard del MIT
%\cleardoublepage
%

%\includepdf[pages=1-3]{dedicatoria2.pdf}


\newpage
\thispagestyle{empty}
\mbox{}


  
\newpage
\thispagestyle{empty}
\vspace*{\fill}
	\begingroup
		\centering
			\begin{left}
				\emph{Me gustar�a dedicar este trabajo de investigaci�n a toda mi familia.}
            \end{left}
	\endgroup
\vspace*{\fill}
\newpage


\newpage
\thispagestyle{empty}
\mbox{}



% The abstractpage environment sets up everything on the page except
% the text itself.  The title and other header material are put at the
% top of the page, and the supervisors are listed at the bottom.  A
% new page is begun both before and after.  Of course, an abstract may
% be more than one page itself.  If you need more control over the
% format of the page, you can use the abstract environment, which puts
% the word "Abstract" at the beginning and single spaces its text.

%% You can either \input (*not* \include) your abstract file, or you can put
%% the text of the abstract directly between the \begin{abstractpage} and
%% \end{abstractpage} commands.

% First copy: start a new page, and save the page number.
\cleardoublepage
% Uncomment the next line if you do NOT want a page number on your
% abstract and acknowledgments pages.
% \pagestle{empty}
\setcounter{savepage}{\thepage}
\begin{abstractpage}
% $Log: abstract.tex,v $
% Revision 1.1  93/05/14  14:56:25  starflt
% Initial revision
% 
% Revision 1.1  90/05/04  10:41:01  lwvanels
% Initial revision
% 
%
%% The text of your abstract and nothing else (other than comments) goes here.
%% It will be single-spaced and the rest of the text that is supposed to go on
%% the abstract page will be generated by the abstractpage environment.  This
%% file should be \input (not \include 'd) from cover.tex.
In this thesis, I designed and implemented a compiler which performs
optimizations that reduce the number of low-level floating point operations
necessary for a specific task; this involves the optimization of chains of
floating point operations as well as the implementation of a ``fixed'' point
data type that allows some floating point operations to simulated with integer
arithmetic.  The source language of the compiler is a subset of C, and the
destination language is assembly language for a micro-floating point CPU.  An
instruction-level simulator of the CPU was written to allow testing of the
code.  A series of test pieces of codes was compiled, both with and without
optimization, to determine how effective these optimizations were.

\end{abstractpage}

% Additional copy: start a new page, and reset the page number.  This way,
% the second copy of the abstract is not counted as separate pages.
% Uncomment the next 6 lines if you need two copies of the abstract
% page.
% \setcounter{page}{\thesavepage}
% \begin{abstractpage}
% % $Log: abstract.tex,v $
% Revision 1.1  93/05/14  14:56:25  starflt
% Initial revision
% 
% Revision 1.1  90/05/04  10:41:01  lwvanels
% Initial revision
% 
%
%% The text of your abstract and nothing else (other than comments) goes here.
%% It will be single-spaced and the rest of the text that is supposed to go on
%% the abstract page will be generated by the abstractpage environment.  This
%% file should be \input (not \include 'd) from cover.tex.
In this thesis, I designed and implemented a compiler which performs
optimizations that reduce the number of low-level floating point operations
necessary for a specific task; this involves the optimization of chains of
floating point operations as well as the implementation of a ``fixed'' point
data type that allows some floating point operations to simulated with integer
arithmetic.  The source language of the compiler is a subset of C, and the
destination language is assembly language for a micro-floating point CPU.  An
instruction-level simulator of the CPU was written to allow testing of the
code.  A series of test pieces of codes was compiled, both with and without
optimization, to determine how effective these optimizations were.

% \end{abstractpage}

\cleardoublepage

\newpage


%\includepdf[pages=1-3]{dedicatoria2.pdf}


\newpage
\thispagestyle{empty}
\mbox{}


  
\newpage
\thispagestyle{empty}
\vspace*{\fill}
	\begingroup
		\centering
			\begin{left}
				\emph{Me gustar�a dedicar este trabajo de investigaci�n a toda mi familia.}
            \end{left}
	\endgroup
\vspace*{\fill}
\newpage


\newpage
\thispagestyle{empty}
\mbox{}

\newpage

%DECLARACION DE ORIGINALIDAD
%\section*{Declaraci�n de originalidad de este
trabajo de \\investigaci�n}

Yo, David Daniel P�rez Sosa, declaro que este
Trabajo Final de Grado titulado 
\emph{
Dise�o del modelo IEC 61850 del regulador de velocidad 
de una unidad generadora t�pica de la 
Central Hidroel�ctrica Itaipu
}
no contiene materiales que han sido 
publicados previamente, en forma completa
o en partes, para la obtenci�n de ning�n 
otro grado acad�mico o diploma. 

Salvo donde indico lo contrario, esta
investigaci�n es fruto de mi propio trabajo.
\\  
\\ 

\par\noindent Autor \dotfill\null\\*
  {\raggedleft David Daniel P�rez Sosa \par}


\newpage

\section*{Agradecimientos}

Agradezco infinitamente a las siguientes personas:

A Dios Todopoderoso, por iluminar mi vida con su amor 
y gracia. 

A mi familia, por ser maravillosamente
buenos conmigo, por apoyarme siempre. A mis padres, 
Teresa y Daniel, les 
agradezco por todo su apoyo y espero
honrrales siempre. A mis hermanitos y hermanitas, 
Diana, Alexis, Noelia, Beatriz, Rub�n, 
y Mar�a Jos� que no est� presente f�sicamente,
por ser los motores que me impulsan 
a crecer como estudiante. 

A mi orientador por parte de la Itaipu, 
el Ingeniero Ladislado Aranda Arriola,
del �rea de ingenier�a de la Itaipu Binacional, 
especialista en la norma IEC 61850, 
quien me ha iniciado en la norma IEC 61850
provey�ndome materiales bibliogr�ficos adecuados,
acceso a las documentaciones que sean necesarias, 
y d�ndome valiosas directrices para 
la elaboraci�n del anteproyecto que 
defini� el rumbo de este trabajo. Muchas 
gracias por confiar en mi persona.

A mi orientador por parte de la Itaipu, 
el Ingeniero 
Rodrigo Andr�s Ramos Galeano, 
del �rea de ingenier�a 
de la Itaipu Binacional, 
especialista en la norma IEC 61850,
por su amistad, y  
por todo el apoyo que me ha ofrecido 
durante la elaboraci�n de este trabajo. 
Sin su ayuda este trabajo no se hubiera hecho 
realidad. Sin duda, su excelencia 
profesional me sirve de inspiraci�n 
en mi formaci�n.

Al Ingeniero Juan Manuel Ramirez Duarte 
mi orientador por parte de la Facultad Polit�cnica,
por incentivarme a estudiar la norma IEC 61850,
por su paciencia  
y todo su apoyo profesional durante la elaboraci�n de 
este trabajo.

Quisiera nombrar de manera especial a la Lic. 
Lidia Benitez de P�rez, decana de la Facultad 
Polit�cnica, y al Dr. Sebasti�n Arce, miembro del Centro de Investigaci�n de la facultad,
quienes han apoyado la realizaci�n de 
trabajo. Gracias a l�deres como la Lic. Lidia y el Dr. Arce, 
que apoyan las investigaciones sobre electrotecnolog�a, 
nuestro pa�s saldr� adelante. 

Esta investigaci�n ha tenido el apoyo, en parte, de la
Fundaci\'on Parque Tecnol\'ogico Itaipu. Muchas 
gracias a todos los directivos 
por haber disponibilizado los 
recursos del PTI, por el viaje al Seminario del 
Sector El�ctrico Paraguayo, al Seminario Nacional de Producci�n y 
Transmisi�n de Energ�a El�ctrica del Brasil, 
y por el acceso al portal de la 
Capes.  

A mi t�a, la Hermana Francisca P�rez, y el 
Ingeniero Hermenegildo Ferreira, 
quienes me han apoyado siempre en mis estudios, 
sin vosotros no hubiera podido llegar a ser ingeniero. Les debo much�simo.

A mis profesores de la Facultad Polit�cnica, 
por todo lo que nos han ense�ado durante 
la carrera, y en especial, a los profesores que 
me han ense�ado los fundamentos te�ricos previos m�s importantes 
para empezar a estudiar la norma IEC 61850 aplicada a Centrales Hidroel�ctricas: 
Ing. Fernandez, profesor de Control y Estabilidad de Sistemas El�ctricos de Potencia,
Ing. F�lix Barrios, profesor de Control y Servomecanismo I y II,
Ing. Charles Santacruz, profesor de Protecci�n, Control y Supervisi�n de Sistemas El�ctricos de Potencia, 
Ing. David Novasky, profesor de Java Avanzado,
Ing. Jos� Coppari, profesor de Programaci�n Orientada a Objetos y XML.


A mis queridos compa�eros y compa�eras:
Cristel, Elena, Fernando, Luis, Arr�a, 
gracias por vuestra amistad, y por 
vuestro aliento durante la elaboraci�n de mi trabajo.

Al Dr. Sidney Viana, quien me ha ense�ado 
aspectos importantes del m�todo cient�fico que he 
aplicado en este trabajo. 

A Leslie Lamport y Donald Knuth, 
al \emph{Massachusetts Institute of Technology}, 
y a @kocolosk de \url{https://github.com/},  
por \LaTeX{}, 
\TeX{}, la clase \emph{mitthesis.cls}, 
y el fork que realiz� del layout 
de tesis de @kocolosk,
gracias 
a ustedes me he divertido poco 
formateando mi trabajo.

A todos los internautas que comparten informaci�n en la red 
de manera altruista. Gracias a todos ustedes 
podemos hacer investigaciones de calidad 
desde Paraguay. Me han
ense�ado que juntos, compartiendo nuestro conocimiento en internet, 
cambiamos el mundo,  
construimos un mundo mejor.  

Al profesor Jos� Coppari, por
haberme permitido asistir como oyente a las clases 
de tecnolog�as Java que tanto me ayudaron
en este trabajo.

A mi primo Rodrigo Saldivar, por haber
aclarado mis dudas sobre redes de computadoras.

A mi colega Sergio Morel, por su valiosa ayuda 
para aprender a programar.

A Daisy, Mabel, Gaby, Luis, Liz, y Juan, por lo bien
que la pasamos mientras elaboraba mi anteproyecto.  

A mis followers de Twitter por los buenos momentos
que pasamos mientra escrib�a mi trabajo.

He dejado de nombrar a muchas personas que me han 
dado su ayuda de una u otra forma 
para la culminaci�n de este trabajo. Expreso 
mi m�s sincera gratitud a todas ellas.










%%%%%%%%%%%%%%%%%%%%%%%%%%%%%%%%%%%%%%%%%%%%%%%%%%%%%%%%%%%%%%%%%%%%%%
% -*-latex-*-
