\section{Detector Response Simulations}

An accurate simulation of the detector response is an important validation of a collaboration's understanding of the apparatus as well as a vital tool in the calculation of some systematic uncertainties. STAR, like many other collider experiments, uses Monte Carlo routines to generate events with properties similar to those measured by the experiment.  These events are collections of identified particles with well-defined kinematics.  STAR primarily relies on the PYTHIA 6 event generator \cite{} in the CDF Tune A \cite{} configuration to simulate proton collisions.  A subset of results from PYTHIA have been validated using the alternative HERWIG event generator \cite{}.  The output from PYTHIA is fed through a GEANT 3.21 \cite{} model of the STAR detector; the end results of this process is a set of detector signals that approximate the actual detector response to a real event.  Significantly, the output from GEANT can be processed by the same reconstruction software used for real data.  Reconstruction efficiencies and other quantities useful for the estimation of systematic uncertainties can calculated by associating particles in the output of the reconstruction software with particles in the ``true'' event record generated by PYTHIA.
