\section{The Simple Parton Model}

% predates QCD, invented to explain scaling of F_{1,2}
% Leader cites 69 Feynman PRL as inventor of parton model, but I skimmed the Letter and I don't see it.
% \cite{Panofsky:1968pb} -- confirmation of Bjorken scaling
% \cite{Bjorken:1968dy} -- Bjorken scaling

In the late nineteen-sixties high energy physicists at SLAC confirmed Bjorken's
hypothesis that the inelastic structure functions of the proton scaled; that is,
at high energies they did not depend on the $Q^2$ of the interaction. This
result stood in sharp contrast to the power-law behavior of the proton's elastic
form factors, and implied the existence of point-like constituents inside the
proton. These ``partons'' were conceived as effectively massless,
electromagnetically charged particles; in the deep-inelastic scattering
(DIS) regime, a virtual photon interacts with a parton, not the proton as a
whole.

Scaling is manifest when Bjorken's $F_1$ structure function is expressed in terms of the number densities $q(x)$ of quarks and $\bar q(x)$ of antiquarks as
%
\begin{equation}
  F_1(x, Q^2) = \frac{1}{2}\sum_{j}{e_j^2[q_j(x) + \bar{q}_j(x)]}
\end{equation}
%
where the sum is taken over quark flavors $j$ and $e_j$ is the electromagnetic charge of flavor $j$.  In longitudinally polarized DIS we define an analogous polarization density $\Delta  q(x) \equiv q_+(x) - q_-(x)$ as the difference in number density between quarks whose spins are aligned with the (longitudinal) spin of the proton and quarks whose spins are anti-aligned; the polarized analogue to $F_1$ is then
%
\begin{equation}
  g_1(x, Q^2) = \frac{1}{2}\sum_{j}{e_j^2[\Delta q_j(x) + \Delta \bar{q}_j(x)]}.
\end{equation}

In the na\"ive parton model we assume $SU(3)_F$ flavor symmetry and thus it is useful to rewrite the expression for $g_1$ in terms of quantities which have specific $SU(3)_F$ transformation properties:
%
\begin{equation}
  g_1(x) = \frac{1}{9}[\frac{3}{4}\Delta q_3(x) + \frac{1}{4}\Delta q_8(x) + \Delta \Sigma(x)]
  \label{eqn:g1}
\end{equation}
%
where
%
\begin{eqnarray}
  \Delta q_3(x) & = & (\Delta u + \Delta \bar{u})_x - (\Delta d + \Delta \bar{d})_x \nonumber \\
  \Delta q_8(x) & = & (\Delta u + \Delta \bar{u})_x + (\Delta d + \Delta \bar{d})_x - 2(\Delta s + \Delta \bar{s})_x \nonumber \\
  \Delta \Sigma(x) & = & (\Delta u + \Delta \bar{u})_x + (\Delta d + \Delta \bar{d})_x + (\Delta s + \Delta \bar{s})_x
  \label{eqn:su3_dis}
\end{eqnarray}
%
The first moments of these quantities are the hadronic matrix elements of an octet of quark $SU(3)_F$ axial-vector currents $J_{5\mu}^i$ and a flavor singlet singlet axial current $J_{5\mu}^0$.  In the limit of massless partons the non-singlet currents are scale-independent quantities, and are known from $\beta$-decay measurements \cite{foobar}:
%
\begin{eqnarray}
  a_3 & = & \int_0^1 dx~\Delta q_3(x) = g_A = 1.2670 \pm 0.0035 \nonumber \\
  a_8 & = & \int_0^1 dx~\Delta q_8(x) = 0.585 \pm 0.025
  \label{eqn:beta-decay}
\end{eqnarray} % might be missing a 1/\sqrt{3} in a_8
%
A measurement of the first moment of $g_1^p$ could thus be interpreted as a measurement of the singlet current $a_0 = \int_0^1 dx~\Delta \Sigma (x)$, and in turn as a measurement of the quark spin contribution to the spin of the proton.  We rearrange \ref{eqn:su3_dis} to yield an expression for the singlet current in terms of $a_8$ and the polarized strange quark densities:
%
\begin{equation}
  \Delta \Sigma = a_8 + 3 \int_0^1 dx~(\Delta s + \Delta \bar s)
  \label{eqn:a_0_prediction}
\end{equation}
%
If one assumes that the sea quark distribution is either unpolarized or CP-symmetric and thus does not contribute to the spin of the proton, Equation \ref{eqn:a_0_prediction} becomes a \textit{prediction} for $\langle S_z \rangle$, as noted by Ellis and Jaffe \cite{Ellis:1973kp} in 1974.

% Should say something about the Sehgal result \cite{Sehgal:1974rz} that concludes the quark contribution to the spin of the proton is $\sim$ 0.3, in agreement with Ellis-Jaffe but using a slightly different derivation (Bjorken sum rule \cite{Bjorken:1966jh}, equivalent parton model result (cites several people), use SU(3) symmetry to obtain same result for $\Xi^- (dss) \rightarrow \Xi^0 (uss) + e + \bar \nu_e$, expresses $\frac{G_A}{G_V}$ ratios in terms of experimentally-measurable quantities $F$ and $D$ (calls that step the octet-model results), seems that $F+D = \frac{G_A}{G_V} \approx g_A = a_3$, F/D 

% In much of the literature I see $g_A$ instead of Griffiths' $\frac{G_A}{G_V}$.  I wonder if that is a definition, or if the literature just takes the Conserved Vector Current hypothesis $G_V = 1$ for granted? -- yes, it's the latter, see Stiegler 1995

% \begin{equation}
%   \Gamma_1^p \equiv \int_0^1 dx~g_1(x) = \frac{1}{9}[\frac{3}{4}a_3 + \frac{1}{4}a_8 + a_0]
% \end{equation}

% \begin{eqnarray*}
%   a_3 \equiv g_A & = & 1.2670 \pm 0.0035 \\
%   a_8 & = & 0.585 \pm 0.025
% \end{eqnarray*}
