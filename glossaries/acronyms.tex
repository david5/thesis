%\newacronym{svm}% label
%{svm}% abbreviation
%{support vector machine}% long form

%\newacronym{ksvm}{ksvm}{kernel support ssssss vector machine XXXXXXXX}

\newacronym{rna}{rna}{-USO interno- ribonukleins\"aure}


\newacronym[description={Statistical pattern recognition 
technique~\protect\cite{svm}}, % acronym's description
name={-USO interno- Support vector machine (svm)}% change the default way
of displaying the entry name in the list of acronyms ]{svm}{svm}{support vector machine}


\newacronym[description={Statistical pattern recognition technique
using the ``kernel trick'' (see also \glshyperlink[SVM]{svm})},% acronym's description
name={-USO interno-Kernel support vector machine (ksvm)}% change the default
way of displaying the entry name in the list of acronyms ]{ksvm}{ksvm}{-USO interno-kernel 
support vector machine}

%%%API
\newacronym[description={Es una interface
a trav�s del cual es posible acceder a 
las funcionalidades de un software}, 
name={API - Application programming interface}]
{API}{API}{Application programming interface}

%%%ACSI
\newacronym[description={El \textbf{ACSI} define los pilares 
del modelo de informaci�n 
y el modelo de servicios
de la norma IEC 61850}, 
name={ACSI - Abstract Communication Service Interface}]
{ACSI}{ACSI}{Abstract Communication Service Interface}


%%%CDC 
\newacronym[description={
Corresponde al nivel 1 del ACSI, a trav�s del cual
se crean los \textbf{Data} de la IEC 61850-7-4 y 
los Data Objects de la IEC 61850-6}, 
name={CDC - Common Data Class}]
{CDC}{CDC}{Common Data Class}


%%%CID
\newacronym[description={SCL file that contains
an instantiated IED within a project, current 
adresses of the IED, and optionally the 
Substation related information related to the IED.
}, 
name={CID - Configured IED Description}]
{CID}{CID}{Configured IED Description}



%%%O-O
% \newacronym[description={Object Oriented should be: Object oriented system, 
% object oriented programming, object oriented model, object oriented 
% technology},
% %name={Intelligent Electronic Device}
% ]{O-O}{O-O}{Object-Oriented}

\newacronym[description={En el texto se habla de sistemas orientados a objetos, 
programaci�n orientada a objetos, modelo orientado a objetos 
y tecnolog�a orientada a objetos. Todas estas definiciones 
se refieren a la aplicaci�n del paradigma orientado a objetos},
name={O-O - Orientado a Objetos}
]{O-O-es}{O-O}{Orientado a Objetos}



%%%ICD
\newacronym[description={Variante SCL que contiene
la descripci�n del IED: La implementaci�n de 
nodos l�gicos, la cantidad de instancias disponibles,
sus servicios y configuraciones de comunicaci�n, 
pero sin contener a�n 
los par�metros de configuraci�n definidos 
por el due�o de la planta}, 
name={ICD - IED Capability Description}]
{ICD}{ICD}{IED Capability Description}


%%%IED
\newacronym[description={Son dispositivos basados en uno o varios 
microprocesadores integrados llamados 
Intelligent Electronic Devices (IEDs). Manejan protocolos de comunicaci�n  
y sistemas de informaci�n con el objetivo de enviar o 
recibir datos de o para varias fuentes para desempe�ar las funciones de  
monitoreo, supervisi�n, control, y protecci�n de la generaci�n, transmisi�n 
y distribuci�n de la energ�a el�ctrica},
name={IED - Intelligent Electronic Device}
]{IED}{IED}{Intelligent 
Electronic Device}


%%%LN
%\newacronym[description={This abbreviation of \emph{Logical Node} is used 
%by IED xml child node in the SCL},%
%name={LN - Logical Node}
%]{LN}{LN}{Logical Node}


%%%LNode
%\newacronym[description={This abbreviation of \emph{Logical Node} is used 
%by Substation xml child nodes in the SCL},%
%name={Logical Node}
%]{LNode}{LNode}{Logical Node}



\newacronym[description={El \emph{Merging Unit} conecta
los transformadores de corriente y potencial convencionales
a los dispositivos de control y protecci�n
a trav�s del bus de proceso normalizado en 
conformidad con la norma IEC61850-9},%
name={MU - Merging Unit}
]{MU}{MU}{Merging Unit}


%%%SAS
\newacronym[description={
Sistema de Automatizaci�n de 
Subestaciones (del ingl�s \emph{Substation Automation System}).
La norma IEC 61850 fue concebida inicialmente para 
subestaciones. Debido a ello, en la norma IEC 
61850 siempre se hace menci�n al SAS. Posteriormente
la norma fue extendida para cubrir las necesidades 
de los sistemas de automatizaci�n de las hidroel�ctricas,
pero debido a que ya se utilizaba extensamente en toda norma la 
palabra ``Subestaci�n'', es correcto 
referirse a la automatizaci�n 
de cualquier parte del sistema de potencia 
como SAS 
},
name={SAS - Sistema de Automatizaci�n de Subestaciones}]
{SAS}{SAS}{Sistema de Automatizaci�n de Subestaciones}

%%%SCD
\newacronym[description={SCL file that contains 
the power system functions, i.e, the substation 
xml node of the SCL, with a high level of detail, 
including descriptions  
such as substation LNodes 
correspondences with IED LN objects, 
all the IED's ICD, headers, 
communications details (identifies adresses and subnetworks) 
},
name={SCD - Substation Capability Description}]
{SCD}{SCD}{Substation Capability Description}


%%%SCL
\newacronym[description={El SCL es un lenguaje basado en 
XML definido como norma (IEC 61850-6 \cite{IEC61850-6:2004})
y es utilizado para la configuraci�n de los 
equipos IEC 61850 de la planta},% acronym's description
name={SCL - Substation Configuration description Language}
]{SCL}{SCL}{Substation Configuration description Language}


%%%SSD
\newacronym[description={Esta variante SCL 
contiene las funciones del sistema de potencia, 
esto es, el elemento \textbf{Substation} del SCL, 
pero sin grandes detalles},
name={SSD - System Specification Description}]
{SSD}{SSD}{System Specification Description}

%%%SO
\newacronym[description={
La serializaci�n, o \emph{marshalling} es el proceso
de persistir un objeto en la memoria o para prop�sitos de transmisi�n,
creando un archivo legible por humanos que contiene la estructura y los
datos en un formato grabable en el computador. Un objeto 
solo existe en tiempo de ejecuci�n, y este es serializado para
ser guardado por el tiempo que sea necesario o
alternativamente, para ser transmitido a trav�s de la red
para otro host},
name={SO - Serializaci�n de objetos}]
{SO}{SO}{Serializaci�n de objetos}


%%%UML
%\newacronym[description={Unified Modeling Language {\texttrademark}
%\cite{UML2:2009} is a specification to 
%describes the model of a application structure, 
%behabiour and system architecture}, %
%name={Unified Modeling Language {\texttrademark} - 
%UML}]{UML}{UML}{Unified Modeling Language}

%%%UML
\newacronym[description={Lenguaje Unificado de Modelado \emph{Unified Modeling
Language} {\texttrademark} \cite{UML2:2009} 
Es una especificaci�n para describir 
el modelo de la estructura de una aplicaci�n, 
comportamiento o arquitectura del sistema}, %
name={UML - Unified Modeling Language {\texttrademark} - 
UML}]{UML-es}{UML}{Lenguaje Unificado de Modelado}


%%%XSD
\newacronym[description={En la capa 1 del modelo OSI
el PDU es un bit, en la capa 2 es una trama, 
en la capa 3 un paquete y en la capa 4 es un segmento},
name={PDU - Protocol Data Unit}]
{PDU}{PDU}{Protocol Data Unit}



%%%XML
\newacronym[description={XML es un formato de texto 
que proporciona un m�todo que facilita a los seres humanos  
la clasificaci�n de nombres de elementos y sus atributos, 
y adem�s, es legible por las m�quinas, existiendo
una gran variedad de tecnolog�as de software 
que la manipulan},
name={XML - eXtensible Markup Language}]
{XML}{XML}{eXtensible Markup Language}


%%%XSD
\newacronym[description={El esquema XML o XSD 
es utilizado para describir la estructura
y las restricciones de los documentos XML. 
En el apartado IEC 61850-6 \cite{IEC61850-6:2004}
se vinculan XSDs a los modelos de objetos serializados
del SCL},
name={XSD - XML Schema Definition}]
{XSD}{XSD}{XML Schema Definition}

