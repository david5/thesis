\section{QCD and the improved Parton Model}

The simple parton model is a heuristic that predates the acceptance of quantum
chromodynamics (QCD) as the theory of the strong interaction. QCD introduces
two important modifications to the parton model:
%
\begin{itemize}
  \item Parton density scaling is violated by a logarithmic $Q^2$ dependence.
  \item $g_1(x,Q^2)$ gains a contribution from the polarized gluon
  distribution in the nucleon.
\end{itemize}
%

Evolution of $g_1$: higher-order diagrams contain collinear divergences,
infinities are handled by factorization which means a choice of factorization
scale. $\mu^2 = Q^2$ is the ``optimal'' choice, as a result parton densities
become $Q^2$-dependent. Dimensional regularization is ``crucial'' technique
for handling collinear (and infrared?) divergences; there are ambiguities in
the application of this technique for the polarized case, leading to
competition between $\bar{MS}$, $AB$, and $JET$ factorization schemes ...
After this, Leader jumps straight into writing down the evolution equations
for the parton density functions.

\begin{equation}
  \alpha_s~ln \frac{Q^2}{m_q^2} = \alpha_s~ln \frac{Q^2}{\mu^2} + \alpha_s~ln \frac{\mu^2}{m_q^2}
\end{equation}

do i need the evolution equations and splitting functions?

gluon contribution to $g_1$:  gluonic version of Adler-Bell-Jackiw triangle diagram 

\begin{equation}
  a_0^{gluons}(Q^2) = -3 \frac{\alpha_s(Q^2)}{2\pi} \int_0^1 dx~\Delta g(x, Q^2)
\end{equation}

or you could write it as

\begin{equation}
  a_0 = \Delta \Sigma - 3 \frac{\alpha_s}{2\pi}\Delta G
\end{equation}

NB: gluon contribution to $a_0$ is zero in $\bar{MS}$ scheme.  Need to use AB or JET scheme to obtain this result.

NB: QCD is invariant under color gauge transformations, but the interpretation of individual Feynman diagrams is gauge-dependent.  The interpretation that ``looks like'' the parton model is obtained by using the light-cone gauge for the gluon vector potential.

More notes on this section from September 28:

QCD corrections to the photon-quark interaction introduce correction terms
that are collinearly divergent. These are renormalizable by factorizing into
hard and soft parts. The optimal choice of a factorization scale is $\mu^2 =
Q^2$, but this means that the (soft) parton densities are now $Q^2$-dependent
and perfect Bjorken scaling is broken. However, the scaling is only
logarithmic and is calculable via evolution equations.

Experiments measure $A \approx A_1 \propto g_1$.  I'm not sure how the EMC experiment establishes that relation between $A_1$ and $g_1$, though.

OK, so we have $Q^2$-dependent parton densities, and if we measure $g_1$ at
multiple values of $Q^2$ for a given $x$ we can solve the evolution equation
for $\Delta G$. When we do this, are we still interpreting $g_1$ using the
simple parton model formula, or do we need to include the anomalous gluon
contribution too?

$g_1(x, Q^2)$ is certainly altered by the $Q^2$ evolution of the parton densities.  It becomes

\begin{equation}
  g_1(x, Q^2) = \frac{1}{2} \sum_{flavors} e_q^2 \left[\Delta q^2 + \Delta \bar{q}^2 + \frac{\alpha_s}{2 \pi} \left(ml\Delta C_q \otimes \Delta q + \Delta C_G \otimes \Delta G\right)\right]
\end{equation}

Depending on the factorization scheme, the first moments might not be altered
by the evolution. In $\bar{MS}$ $a_3$ and $a_8$ are invariant, in $AB$ $\Delta
\Sigma$ is invariant, and in $JET$ all three are. But, just because they are
independent of $Q^2$ does not mean they keep their old definitions in terms of
the first moments of parton densities, does it? I don't know.

The Adler, Bell, Jackiw triangle diagram introduces an anomalous gluonic
contribution to $a_0$, and thus to the first moment of $g_1$. This is
completely separate from the $Q^2$ evolution of the parton densities. It means
that a small measured value of $a_0$ does not necessarily imply small quark
polarization in the proton (this interpretation only valid in $AB$ and $JET$,
because $\Delta \Sigma$ varies with $Q^2$ in $\bar{MS}$ and thus it is not
appropriate to consider it as a spin).

So the question becomes, if I were to take the first moment of $g_1(x)$ as
written above, would I get $\frac{1}{9}\left[\frac{3}{4}a_3+\frac{1}{4}a_8 +
a_0\right]$, where $a_0$ now includes the anomalous gluonic term? Can it be
that simple?

$\Delta C_{q,G}$ are ``Wilson coefficients evaluated from the hard part calculated beyond the Born approximation'' (whatever that means).
