
\section{Adressing and routing}

Each node must be able to say which of the other nodes on the 
network it wants to communicate with. This is done by assigning
an address to each node. An address is a byte string that identifies a node;
that is, the network can use a node’s address to distinguish it from the other
nodes connected to the network. When a source node wants the network to
deliver a message to a certain destination node, it specifies the address of
the destination node. If the sending and receiving nodes are not directly
connected, then the switches and routers of the network use this address to
decide how to forward the message toward the destination. The process of
determining systematically how to forward messages toward the destination node
based on its address is called routing \cite{PetersonDavie:2003}.

Adressing types: 

	\subsection{Unicast}
	The source node wants to send a message to a single destination node.
	
	\subsection{Broadcast}
	The source node want to \emph{broadcast} a message to all the 
	nodes on the network.
	
	\subsection{Multicast}
	The source node send a message to some subset of the 
	other nodes, but not all of them.
	

	
	
	
