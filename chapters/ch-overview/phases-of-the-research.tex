\section{Phases of the research}

This research process follow the steps mentioned here 
in the respective order: 

\todo{traducir al ingles}
\todo[inline]{Esto lo quite de mi anteproyecto, debo pulirlo bastante,
las fechas estan mal, debo colocar las fechas indicadas 
en mi anteproyecto aprobado. Yo reaproveche este texto 
que ya habia redactado en \LaTeX{} a finales del a\~no pasado.}

	\subsection{Object oriented paradigm}
		\emph{Agosto del 2009}
			Clases, m\'etodos, atributos. 
			Objeto. 
			Instanciaci\'on.
			Herencia. 
			Abstracci\'on. 
			Encapsulaci\'on. 
			Polimorfismo. 
			Interfaces. 
			Recursividad. 
			Tipos de datos.
	\subsection{Septiembe del 2009}
		\emph{Unified Modeling Language - UML}:
			Diagrama de clases. 
			Representaci\'on de objetos mediante UML.
	\subsection{Octubre del 2009}
		\emph{XML - Extensible Markup Language}:
			Lenguajes de marcaci\'on.
			XML.
			DTD (\emph{Document Type Definition}).
			XSD (\emph{XML Schema Definition})
	\subsection{Noviembre del 2009}
		\subsubsection{An\'alisis de los 
		conceptos de la automatizaci\'on 
		de Sistemas El\'ectricos de Potencia}
			Sistema de automatizaci\'on 
			de hidroel\'ectricas, 
			en especial de una unidad generadora. 
			Topolog\'ia de red del 
			sistema de automatizaci\'on. 
			Funciones del sistema de automatizaci\'on 
			de una unidad generadora: comando, 
			adquisici\'on de datos, 
			protecciones, supervisi\'on, 
			alarmas, secuencia de eventos, 
			enclavamientos y bloqueos.
		\subsubsection{Identificaci\'on de las 
		comunicaciones en el Sistema de Automatizaci\'on El\'ectrico}
			Redes de \'area local.
			Tecnolog\'ias de red, en especial las 
			implementaciones de la norma IEEE 802.3. 
			Jerarqu\'ias de protocolos.
			Servicios: Connection-Oriented y Connectionless.
			Relaciones entre servicios y protocolos.
			Modelo de referencia \emph{Open System Interconnection} (OSI). 
			Medios de transmisi\'on de informaci\'on. 
			Control, flujo, correcci\'on y detecci\'on de errores. 
			Algoritmos de enrutamiento: broadcast y multicast.
			Arquitecturas cliente-servidor, \emph{peer-to-peer}.
	\subsection{Diciembre del 2009 y enero del 2010}	
		\emph{Lectura e interpretaci\'on de 
		la norma IEC  61850}: Incluyendo las 
		partes 1, 2, 4 (solo la sub 
		parte 5), 6, 7-1, 7-2, 7-3, 7-4.
	\subsection{Febrero del 2010}
			{Desglosar la arquitectura, 
			los elementos y modelos de comunicaci\'on 
			de un Sistema de Automatizaci\'on de Usinas}
	\subsection{Marzo del 2010}
		\emph{Conocer el funcionamiento y clasificar las 
		funcionalidades  generales necesarias en el 
		sistema de automatizaci\'on de un regulador 
		de velocidad}: Comandos, adquisici\'on de datos, 
		protecciones, supervisi\'on, alarmas, 
		secuencia de eventos, enclavamientos, 
		secuencias autom\'aticas, controles de 
		velocidad, sincronizaci\'on, informes, 
		valores hidroenerg\'eticos, entre otros.
	\subsection{Abril del 2010}
		\emph{Estudio del funcionamiento y 
		caracter\'isticas particulares del 
		regulador de velocidad de una unidad 
		generadora t\'ipica de Itaipu}: 
		Identificaci\'on de las funciones de 
		automatizaci\'on con ayuda de especialistas 
		de la m\'aquina. Desglosar cada 
		funci\'on de automatizaci\'on en los 
		nodos l\'ogicos correspondientes.
	\subsection{Mayo del 2010}
		Breve estudio de la arquitectura de red 
		necesaria para la automatizaci\'on del 
		regulador de velocidad implementando la norma IEC 61850.
	\subsection{Junio y Julio del 2010}
		Modelado de los nodos l\'ogicos 
		normalizados del apartado 
		IEC 61850-7-4-10 \emph{Hydro Power Plants} utilizando 
		herramientas de ingenier\'ia disponibles en el mercado.
	\subsection{Agosto a Septiembre del 2010}
		Identificaci\'on y modelado de los servicios 
		de comunicaci\'on necesarios para el regulador 
		de velocidad. Apartado 7-2 de la norma IEC 61850.
	\subsection{Octubre a Diciembre del 2010}
			Validaci\'on y correcci\'on de errores del 
			trabajo mediante simulaci\'on por software 
			en una red IEC 61850 y posterior revisi\'on por pares.

