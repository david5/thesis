\section{Introduction}

Energy worldwide is at least a seven trillion dollar
a year business and expanding. Energy profoundly affects 
our economy, society, and environment \cite{Dukert:2009ab}, 
then, they are a continously researching about new technologies 
and standards in the electrical power systems, in areas such as 
information automation, related cyber security and data 
transfer performance through Smart Grids.

La energ\'ia en todo el mundo mueve por lo menos 7 billones de 
d\'olares anuales, y es un negocio en expansi\'on. Por ello, la 
energ\'ia afecta profundamente nuestra econom\'ia, sociedad y 
entorno \cite{Dukert2009}, debido a esto, existe una constante 
investigaci\'on, y han surgido nuevas tecnolog\'ias y est\'andares 
en los sistemas de potencia, en \'areas como la automatizaci\'on 
y tratamiento de la informaci\'on, por dar unos ejemplos, con el
objetivo de mejorar el performance y la seguridad en el sistema 
de potencia, a trav\'es de redes inteligentes (Smart Grids).

In the lasts years,  the power systems automation over the world
uses microprocessors embedded devices 
\cite{Santoso:2000, Schwarz:2000mc} called
Intelligent Electronic Devices (IEDs) that suport 
network communication technologies with
the aim of to send or receive information from or to many sources
to monitoring, control and supervise the generation, transmission 
and distribution of the energy. 

En los \'ultimos a\~nos, la automatizaci\'on de sistemas de potencia 
en todo el mundo utiliza ampliamente dispositivos basados en uno o 
varios microprocesadores \cite{Santoso:2000, Schwarz:2000mc} integrados 
llamados Intelligent Electronic Devices (IEDs)  que utilizan 
tecnolog\'ias de redes de comunicaci\'on con el objetivo de enviar 
o recibir informaci\'on de o para varias fuentes con el objetivo 
de monitorear, controlar, y supervisar la generaci\'on, transmisi\'on 
y distribuci\'on de la energ\'ia
   \cite{McDonald:2007, IEEE:1997dic, Schwarz:2008wi}.

The information interchagebility between IEDs from diferent
vendors was coming complex, expensive, and sometimes 
impossible, then the industry 

\ldots\ldots\ldots

to adopt the IEC 61850 standard to achieve the interoperability, 
realiability and more quality for the information interchange 
of the Energy Management System (EMS).

\ldots\ldots\ldots aca debo colocar las siglas de EMS, ver como
arreglar este tema para que quede todo bien el tema de las siglas
  

El intercambio de informaci\'on entre IEDs de diferentes fabricantes 
se ha vuelto muy complejo, costoso, y a veces imposible, es por ello 
que la industria se ha puesto de acuerdo para adoptar el est\'andar 
IEC 61850 y as\'i conseguir interoperabilidad, confiabilidad y mayor 
calidad en el intercambio de informaci\'on dentro del Sistema de 
Gerenciamiento de Energ\'ia (EMS-Energy Management System). 

The IEC 61850 standard ``Comunication Networks and Systems in Substations" 
provide a support for sustainable interoperability between IEDs: 
information model, information interchange methods, communications 
protocols mappings and a Substation Configuration Language (SCL) 
for electrical energy systems (Generation, Transmission and Distribution) 
for hight, might and low voltage \cite{Schwarz2008}. 

La norma IEC 61850 ``Comunication Networks and Systems in Substations" 
provee un perfecto soporte para una interoperabilidad sustentable entre 
IEDs: modelado de la informaci\'on, m\'etodos para intercambio de la 
informaci\'on, mapeo a protocolos de comunicaci\'on, y un lenguaje de 
configuraci\'on de subestaciones (SCL) para sistemas el\'ectricos de 
energ\'ia (Generaci\'on, Transmisi\'on y Distribuci\'on para alta, 
media y baja tensi\'on) \cite{Schwarz2008}. 

Initially, the standard IEC 61850 focuses on substations, 
and now it are extended to 

\ldots\ldots\ldots satisfacer no se decir..

the totality of the 

La norma IEC 61850, en la actualidad, no se enfoca \'unicamente a 
subestaciones, tambi\'en es aplicable y extensible para satisfacer 
las necesidades de casi la totalidad de la cadena de suministro de 
energ\'ia, entre los cuales destacamos la protecci\'on de l\'ineas de 
transmisi\'on, plantas de energ\'ia e\'olica, distribuci\'on de 
energ\'ia y centrales hidroel\'ectricas, sistemas fotovoltaicos y 
coches el\'ectricos \cite{Schwarz2005, DER2009, German2009}. 

El modelado jer\'arquico de la informaci\'on a trav\'es de nodos 
l\'ogicos es una cuesti\'on clave. La agrupaci\'on correcta de los 
nodos l\'ogicos representan funciones o equipos utilizados en los 
sistemas de potencia. Cada nodo l\'ogico provee una lista de informaci\'on 
bien designada y organizada. Los objetos y servicios definidos en la parte 
IEC 61850-7-2 de la norma permiten el intercambio de esta 
informaci\'on \cite{TC572004}.

En julio del 2007 las extensiones de los nodos l\'ogicos a centrales 
hidroel\'ectricas han sido aprobadas, publicadas y est\'an listas 
para su uso, en el apartado IEC 61850-7-4-10: 
\emph{Hydroelectric Power Plants - Communication for monitoring and control}; 
agregando 60 nodos l\'ogicos y 350 \emph{Data Objects} a la serie 
IEC 61850 \cite{IEC61850TC57, Schwarz2008b}.

Este trabajo consiste en la aplicaci\'on de la norma IEC 61850, 
en especial del modelado de nodos l\'ogicos definidos en la parte 
7-4-10 Hydro Power Plants y de los objetos y servicios de comunicaci\'on 
para la automatizaci\'on de una unidad generadora t\'ipica de Itaipu, 
y proponer al TC57 (International Electrotechnical Commission, Technical 
Committee 57) la complementaci\'on o extensi\'on de nodos l\'ogicos de 
la norma que actualmente son insuficientes para las unidades generadoras 
de Itaipu. Como estudio de caso, se modelar\'an los nodos l\'ogicos y 
servicios de comunicaci\'on necesarios para el regulador de velocidad de 
la unidad generadora. Este trabajo de investigaci\'on se basa en el \'item 
del documento ``Proposta de Temas para Monografias de 
Especializa\c c\~ao - Automa\c c\~ao, Controle e Supervis\~ao do 
Processo el\'etrico Baseado na Norma  IEC 61850 -  A-4 - Automa\c c\~ao 
de Unidades Geradoras - Modelagem Completa da Unidade Geradora" de la 
Itaipu Binacional, redactado por Marcos Fonseca Mendes, Antonio Sertich 
Koehler, Ladislao Aranda Arriola, funcionarios de la Itaipu Binacional. 
