\section{Organization of this Research Document}
This text is organized as follows:
For comming\ldots

\todo{traducir al ingles - organizacion de la tesis}

\begin{itemize}
  \item Chapter 1: revisi�n de la tesis 
  \item Chapter 2: \ldots
\end{itemize}

\todo[inline]{
		\emph{Cap�tulo: }hacer una revisi�n de la tesis
		
		\emph{capitulo: } 	capitulo introductorio a la norma IEC 61850,
						al estilo de la parte 1 y la parte 7-1, sin 
						entrar en puntos muy tecnicos, solo para dar 
						una nocion de lo que trata la norma, 
						asi como los famosos powerpoints que te 
						presentan para la norma y de dan una 
						nocion super super breve sobre que trata 
						la norma.
		
		\emph{Cap�tulo: }	hablar sobre programaci�n orientada 
						a objetos, uml y xml
						demasiado bien va a caer comenzar con
						este cap�tulo, dado que mi tema trata
						sobre modelado de objetos.
		
		\emph{Cap�tulo: } 	hablar sobre subestaciones, lo b�sico 
				para poder entender la norma. 
		
		\emph{Cap�tulo: } 		hablar sobre la automatizaci�n de usinas
				(basado en la tesis brasilera que me paso el Ing. Aranda)
		
		\emph{Cap�tulo: } 	hablar sobre redes, de absolutamente todos
				los conceptos necesarios.
		
		\emph{Cap�tulo: }	se puede estandarizar a UML las notaciones
				y explicaciones de arquitecturas que
				explica la norma. Posible contribucion: utilizar 
				OCL para mejorar la especificacion.
		
		\emph{Cap�tulo: }	hablar sobre la turbina itaip�
		
		\emph{Cap�tulo: }	hablar sobre la norma en si.
				(esto se puede subdividir otra vez
				en m�s cap�tulos especializados si es
				necesario)
		
		\emph{Cap�tulo: } 	se puede hablar sobre las funciones 
				de automatizaci�n aplicadas a la turbina 
				(o al regulador directamente) \\
		
		y los siguientes cap�tulos\ldots veremos despu�s\ldots
		tambien es probable de que varios de estos futuros 
		capitulos se fusionen en uno.
		
		
}


\todo[inline, color=blue!50!red!50]{hay capitulos que 
ya tengo bien aprendido, pero aun no escribi, en las 
vacaciones de invierno voy a escribir bien sobre redes
y las demas teorias basicas de automatizacion. Por 
el momento, he creado mapas mentales que apareceran 
en los capitulos inconclusos que ya tengo aprendido
pero solo me falta escribir. Para que pueda escribir, 
esos capitulos con un buen nivel tecnico voy a 
apoyarme en papers de varias asociaciones, principalmente
el de la IEEE. Aun no estoy pudiendo suscribirme y 
desde la CAPES no se accede, apenas pueda me suscribire 
y escribire esos capitulos. Siempre mi material 
principal es la norma en si, y en base a eso ya
tengo varios capitulos en mente (aprendidos 
pero por documentar) o ya escritos.}


