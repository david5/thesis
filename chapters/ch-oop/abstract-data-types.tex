\subsection{Abstract data types}

Abstraction and information hidding form 
the foundation of all object-oriented 
development \cite{Levy:1984}. 
An abstraction is a simplified description, 
or specification, of a system that emphasizes 
some of the system's details or properties while
suppressing others \cite{Shaw:1984}.
Information hidding, as first promoted by Parnas,
\todo{explicar mas c/mis palabras} 
goes on to suggest that we should decompose 
systems based upon the principle of hidding 
design decisions about our 
abstractions \cite{Parnas:1979} \cite{Grady:1995}.

The abstraction and information hidding 
are very common in electrical equipments and 
mathematical representations of the 
electrical world. The 
models are abstracted 
and is possible to identify the object and 
operations that exist at each level of integrations 
Thus, 
when \todo[inline]{ver si las
palabras estan bien escritas
en este ejemplo} we work 
with phasors to represent a current, which 
leave just the static amplitude and phase
information. The time space are hidden with 
the purpose to manage the information 
at a more hight level, 
thereby skipping the trigonometric calculations 
derived from the time dependence of the sine wave, 
and the information are combined just algebraically, 
simplifying certain kinds of complex 
calculations.  \cite{Grady:1995}

%repetido!
%The abstraction  and information hidding are common 
%in our activities, we abstract the models by 
%identify the object and operations that exist 
%at each level of integration. Thus, when
%\todo[inline]{ver si las palabras estan 
%bien escritas en este ejemplo} 
%working a transformer, we consider the 
%taps, the current on the low and hight side, 
%the transformation relation \cite{Grady:1995}.

The use of abstract data types on a object oriented system 
help to a more precise and at the same time simple on the 
specification taking adventage of the information hidding 
provided by abstract 
data types. \cite{} \todo{aca debo citar mi paper sobre ACSI}


		\lstinputlisting[label=codeGooseWithAbstractA,
		caption=Goose Abstract Class with attributes, 
		abstract methods, getters and
		setters in Java] {chapters/ch-oop/source/java/src/GOOSE.java}


