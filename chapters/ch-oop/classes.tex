\section{Classes}

A class is a static 
%off-line \todo{es realmente off-line?}
template from which objects are 
created. Classes are used 
to classify objects thanks 
that its defines 
common operations 
and a data structure, 
and the types of 
datas that the object can 
store.\\


%Codigo fuente
%C:\Documents and Settings\DELL\Mis
%documentos\tesismayo\tesismayo\thesis\chapters\ch-oop\source\java\src\HelloWorld.java
%\lstinputlisting[label=samplecode,caption=A sample]{sourceCode/HelloWorld.java}
%chapters\ch-oop\source\java\src\HelloWorld.java
\lstinputlisting[label=codeClass,
caption=Class in Java]{chapters/ch-oop/source/java/src/SERVER_v1.java}

%%TODO: cite adobe book

\subsection{Attributes}
\todo[inline]{completar esta parte}
	\lstinputlisting[label=codeAttributes,
	caption=Class with attributes in Java]
	{chapters/ch-oop/source/java/src/SERVER_v2.java}



\subsection{Methods}
Methods are functions that are part of a class 
definition. Once an instance of the class is created, 
a method is bound to that instance.\\
	\lstinputlisting[label=codeMethod,
	caption=Class with attributes and methods in Java]
	{chapters/ch-oop/source/java/src/SERVER_v3.java}



	\subsubsection{Get and set accessor methods}
	Get and set accessor functions, also called getters 
	and setters, allow you to adhere to the programming principles of 
	information hiding and encapsulation while providing an 
	easy-to-use programming interface for the classes that you 
	create. Get and set functions allow you to keep your class 
	properties private to the class, but allow users of your class 
	to access those properties as if they were accessing a 
	class variable instead of calling a class method. 
	The advantage of this approach is that you can avoid 
	having two public-facing functions for each property 
	that allows both read and write access. \\

		\lstinputlisting[label=codeMethod,
		caption=Class with attributes, methods, getters and setters in Java]
		{chapters/ch-oop/source/java/src/SERVER_v4.java}
	
	
	\subsubsection{Constructor methods}
	Constructor methods, sometimes simply called constructors, 
	are functions that share the same name as the class in 
	which they are defined. Any code that you include in 
	a constructor method is executed whenever an instance of the 
	class is created with the  new  keyword. \\

		\lstinputlisting[label=codeMethod,
		caption=Class with attributes, methods, 
		getters, setters and constructors in
		Java] {chapters/ch-oop/source/java/src/SERVER_v5.java}

	