\subsection{Important concepts and terms}

%TODO: add citation from adobe book
%Aca va la pagina 99 de actionscript programming.
The following reference list contains important 
oriented-objects programming terms 
that you will encounter in the IEC 61850 standard. 
%citar las prácticas comunes
\begin{itemize}
  \item Attribute: A characteristic assigned to a class 
  element (such as a property or method) in the class definition.
  \item Attributes are commonly used to define whether 
  the property or method will be available for access by 
  code in other parts of the program. For example, 
  private and public are attributes. A private method 
  can be called only by code within the class, while a public 
  method can be called by any code in the program.
  \item Class: The definition of the structure and behavior 
  of objects of a certain type (like a template or 
  blueprint for objects of that data type).
  \item Class hierarchy: The structure of multiple 
  related classes, specifying which classes inherit 
  functionality from other classes.
  \item Constructor: A special method you can define in 
  a class, which is called when an instance of the 
  class is created. A constructor is commonly used to 
  specify default values or otherwise perform setup 
  operations for the object.
  \item Data type: The type of information that a 
  particular variable can store. In general, data 
  type means the same thing as class.
  \item Dot operator: The period sign ( . ), which 
  in many programming languages) is used to 
  indicate that a name refers 
  to a child element of an object (such as a 
  property or method). For instance, in the expression 
  myObject.myProperty, the dot operator indicates 
  that the term  myProperty is referring to some 
  value that is an element of the object named myObject.
  \item Enumeration: A set of related constant 
  values, grouped together for convenience as properties 
  of a single class.
  \item Inheritance: The OOP mechanism that allows one 
  class definition to include all the functionality 
  of a different class definition (and generally 
  add to that functionality).
  \item Instance: An actual object created in a program.
  \item Namespace: Essentially a custom attribute, 
  allowing more refined control over which code can 
  access other code.   
\end{itemize}

