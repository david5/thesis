\section{Introduction}

The IEC 61850 information model 
is classified as a  
object-oriented (O-O) system. Is necessary 
to have a clear understanding of O-O technology
to dive into IEC 61850 object modelling. 
For this reason this chapter describes 
the object-oriented technology
on which the IEC 61850 information model 
has been standarized\todo[inline]{No se si esta 
bien escrita la palabra standarized}. 

The chapter does not describe all the O-O principles, 
just \todo{�trans:just focuses or focuses just?} focuses 
on the neccesary
background to understand the IEC 61850 
information model. 
The sections which follow will define the 
background which the IEC 61850 
object oriented information system 
was constructed, both formally 
and through examples.

A detailed description of the O-O fundamentals  
and reference materials are provided. To achieve a 
robust concepts comprehension a practical 
implementation with the Unified Modeling Language (UML) 
and Java are provided. This chapter is destined 
for electrical engineers and professionals of 
related areas which the O-O software construction 
is not part of the curriculum. For this reason, 
the practical examples uses electricity area 
elements. Computer science professionals 
could skip this chapter.



\todo[inline]{Me falta explicar que 
los ingenieros electricos 
generalmente no tienen base en programaci�n orientada
a objetos, por eso decid� dedicar todo un cap�tulo
para O-O systems. Sin un robusto conocimiento de
esta etapa es casi imposible conocer a profundidad
el modelo de informaci�n de la norma}
