\section{The objects relationships}

\todo[inline, color=blue!50!red!50]{esto no va en la
seccion de objetos, pues antes necesite explicar 
la secci\'on \ref{sec:Diference-betwen-Classes-and-Objects} 
que trata sobre la diferencia entre objeto y clase. 
Esto es importante como una previa para esta seccion dado 
que las subsecciones a continuacion se presentan como 
clases pero describen las relaciones 
de los objetos. por eso es importante entender bien la 
diferencia primero. Esto es por el lector, para que 
le sea mas facil entender.}


	\subsection{Association}
		\todo{completar - association}
	
	\subsection{Aggregation}
		\todo{completar - aggregation}
		
	\subsection{Composition}
		\todo{completar - composition}
		
	\subsection{Other relationships}
		\todo[inline, color=blue!50!red!50]{por el momento 
		aca solo hay uno, por eso uso una subsubsection, 
		pero es altamente probable que coloque mas cosas
		si es que encuentro que son utilizados en la norma.}
		\subsubsection{Dependency}
			\todo{completar - dependency: la flechita del uml}
