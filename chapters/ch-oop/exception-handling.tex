\section{Exception handling}

%brevemente inspirado en este link
%http://www.allinterview.com/showanswers/59519.html
The exception is an unwanted and 
unexpeceted event which occours 
at runtime, and are caused by an unusual 
situation of the software that  
stop the normal secuence of  
software operations. 

When a exception event are 
dispatched by the system, 
the error could be handled 
by a specific modelled object 
if the software developer anticipated  
it as a posibility. 

The exception could be modelled 
with different handling stategies,     
\cite{MoonStallman:1983}
\cite{Dony:1988}
\cite{DonyC:1990}
\cite{Leavens:1991}
\todo[inline]{buscar mas papers de referencia 
			para colocar aqui por si alguien 
			tenga interes de profundizar en esto
			(buscar en la acm) }
%for example, could be represented 
%by classes of which 
%concrete exceptions would be some
%referenceable instances.  
for example, could be represented 
by classes specifically 
designed for this purpose of which 
concrete exceptions would be some
referenceable instances that 
handles the event of error. 

%\begin{comment}
%	\todo[inline]{leer la siguiente bibliografia:
%	
%	[MSW83] David Moon, Richard M. Stallman, and  Daniel Weinreb.  Lisp Machine 
%	Manual  (fifth edition). Massachusetts Institute of Technology, Artificial
%	Intelligence Laboratory, Cambridge, Mass., January  1983.
%	
%	5.4 Exception Handling
%	Hierarchies can also be used to classify 
%	and organize exceptions in large 
%	software systems. I think this was
%	first done in the Flavors mechanism 
%	of the Lisp Machine [MSW83]. Recent 
%	papers on this topic include
%	[Don88] [Don90].
%	
%	Extraido del paper: 
%	introduction to the literature on O-O design, 
%	programming and languages.
%	Gary t. leavens
%	}

%esta tambien es una buena partida para saber los distintos
%tipos de errores que no trato aqui por ser innecesarios
%para la norma
%http://www.sap-img.com/abap/difference-between-error-and-exception.htm
%\end{comment}



	\lstinputlisting[label=codeMethod,
	caption=Service Error Class in Java]
	{chapters/ch-oop/source/java/src/ServiceError.java}

