\section{Funciones del Sistema de Automatizaci�n de Subestaciones}

Las funciones del \gls{SAS} son las tareas que 
tienen que ser realizadas
dentro de la hidroel�ctrica.

En las centrales hidroel�ctricas existen funciones 
para monitorear, controlar y proteger 
los equipamientos del sistema. Tambi�n 
existen funciones para la configuraci�n del sistema,
gerenciamiento de la comunicaci�n,
o gerenciamiento del software. \cite{IEC61850-5:2003}.



\subsection{Libre ubicaci�n de funciones}

	Desde hace m�s de 20 a�os han 
	surgido las necesidades (y se comenz� a investigar al respecto)
	de tener la libertad de distribuir efectivamente
	en diferentes equipamientos 
	los objetos (virtuales) que realizan 
	tareas computacionales.
	
	Estas frases fueron extraidas de la investigaci�n
	de Jazayeri, de 1988 \cite{Jazayeri:1988}:


%	The effective distribution of Logical Nodes 
%	on diferents IEDs  
%	is a reality thanks to researches about 
%	the structure of distributed systems. More 
%	than 20 years ago emerged requirements 
%	for the object paradigm to suport the 
%	design and development of distributed systems.
	
%	Theses quotes were extracted from Jazayeri 
%	1988 research:
	
	\emph{
	``An object on one node can send a (multicast) message 
	to several other objects \ldots''
	} \cite{Jazayeri:1988} (Hoy en d�a esto es una realidad a trav�s de 
	las asociaciones \textbf{MCAA} y los mensajes \textbf{GSE}
	definidos en las clases \textbf{ACSI})
	
	\emph{
	`` \ldots The ability to group 
	a set of objects and address them as one entity 
	is important in many applications both from an 
	efficiency point of view and from a program 
	structuring point of view \ldots'' 
	} \cite{Jazayeri:1988} (este aspecto tambi�n 
	es una realidad, en la norma 
	IEC 61850 se utilizan \textbf{DATA--SETs} \cite{Ozansoy:2009b}, 
	\textbf{FCD},
	\textbf{FCDA}, entre otros).
	
	\emph{
	`` \ldots a final 
	difference is that our objects are active and 
	not reactive, in the sense that they can start 
	up spontaneosly performing operations, not 
	necessarily only in response to method invocations.
	Such a facility is useful, for example, to allow objects 
	to monitor the enviroment and change their behavior based 
	on changes in the enviroment \ldots'' 
	} \cite{Jazayeri:1988} (En el contexto de la norma IEC 61850
	los objetos activos ser�an las clases abstractas
	(utilizadas como interfaz para el env�o de los mensajes)
	\textbf{URCB}, \textbf{BRCB}  a trav�z de 
	sus opciones de \textbf{trigger}) \cite{Jazayeri:1988}.
	 
%	Some active objects 
%	are GOOSE, URCB and some passive object 
%	are \todo{completar y ver si esta bien})
	
	
	
	Las especificaciones de la norma IEC 61850,
	basadas en
	tecnolog�as orientadas a objetos ya maduras,
	consiguen atender 
	los requisitos de disponibilidad,
	la filosof�a de la central hidroel�ctrica,
	requisitos de performance,
	costos,
	y el estado del arte de la tecnolog�a. 
	
	
\subsection{Clasificaci�n de las funciones seg�n los niveles}
	Seg�n los niveles en el cual el IED se desempe�a,
	sus funciones pueden ser clasificadas en 3 grupos: 
	
	\begin{itemize}
	  \item Nivel de proceso: I/O remotas, actuadores, sensores.
	  \item Nivel de bay: IEDs de control, monitoreo y protecci�n
	  \item Nivel de estaci�n: Estaci�n de ingenier�a de la subestaci�n,
	  base de datos, interfaces para comunicaci�n remota.  
	\end{itemize}
	
	
