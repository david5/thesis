\section{Estado del arte del proceso de ingenier�a IEC 61850}
	\label{sec:ENFOQUE-estado-del-arte-herramientas}

Durante la experiencia adquirida 
con el uso de herramientas de ingenier�a IEC 61850 \cite{PTI:SESEP2010}, 
el autor ha observado que 
el uso de estas herramientas como apoyo a la elaboraci�n
de especificaciones y proyectos ejecutivos es un campo muy nuevo en el �rea de
la ingenier�a el�ctrica. Todas las herramientas disponibles en el mercado se
encuentran en sus primeras versiones estables, y existen muchas teor�as,
discusiones y debates vigentes respecto de las caracter�sticas necesarias en
una herramienta de este tipo para soportar el proceso de ingenier�a IEC 61850.       

Las diversas herramientas tercerizadas existentes en el mercado implementan lo
establecido abstractamente por la IEC61850 usando metodolog�as y flujos de
trabajo diferentes en mayor o menor grado, en ocasiones con interpretaciones
diferentes de lo establecido por la norma. Esto es natural, considerando la
complejidad de la misma y que estas son las primeras implementaciones pr�cticas
disponibles hoy en d�a como productos comerciales. 

Las herramientas Helinks y Visual SCL
proponen utilizar flujos de trabajo similares. Dichas 
herramientas han sido dise�adas para asistir el proceso de
ingenier�a denominado \emph{``Top down engineering''}, 
que consiste b�sicamente en la especificaci�n del sistema 
a trav�s de la variante SSD, la selecci�n e inclusi�n 
de los ICDs que realizen las funciones definidas en el SSD,
y la configuraci�n completa del sistema a trav�s de 
la variante SCD donde se realizan referencias cruzadas 
entre los archivos ICD y el SSD.

La herramienta H\&S STS fue dise�ada para trabajar con 
el mismo flujo de trabajo que las dos herramientas arriba 
mencionadas, pero como paso alternativo, propone
que el proyectista dise�e los 
de ICDs en modo gr�fico a partir del SSD, 
para especificar los IEDs (con sus ICDs)
a ser comprados para el proyecto. De hecho, esto tambi�n puede
ser realizado en las herramientas anteriores, pero no fueron 
optimizadas para ello, por lo tanto, no dan mucha asistencia 
al proyectista.   

La herramienta Kalkitech SCL Manager tambi�n propone el mismo 
flujo de trabajo que las dos primeras herramientas mencionadas,
y adem�s tiene un m�dulo para crear variantes ICD. 

La herramienta de ingenier�a Atlan61850 propone iniciar 
el proceso de ingenier�a con la inclusi�n de los ICDs 
en el proyecto, omitiendo el uso de la variante SSD, 
y creando directamente los CIDs
correspondientes. A este enfoque se lo denomina
\emph{``Bottom up engineering''}.

El autor de este trabajo ha analizado las posiblidades de 
utilizar estas herramientas durante el proceso de ingenier�a,
pero debido a las dificultades encontradas (Helinks no posee 
el nodo l�gico \textbf{KTNK}, en la versi�n demo de Kalkitech SCL Manager
no es posible guardar el proyecto, en la versi�n demo de H\&S STS
se habilitan menos de una decena de nodos l�gicos de subestaciones,
y la herramienta Atlan61850 acepta los ICDs y XSDs desarrollados por el 
autor pero requiere la profundidad del ICD hasta los \textbf{DAI}
para manejar todos los puntos del sistema)
el autor no ha utilizado una sola herramienta, 
sino que utiliza un enfoque mixto 
usando varias herramientas para definir un nuevo enfoque, 
adecuado al problema planteado, que ser� explicado en este cap�tulo.

