\section{Consideraciones para construir sistemas a prueba de futuro}

La norma IEC 61850 abre las posibilidades al proyectista para que este 
pueda construir sistemas a prueba de futuro. Si bien este no es uno 
de los objetivos fundamentales de la norma IEC 61850, 
si el proyectista lo desea, puede considerar ciertos aspectos t�cnicos 
que permitan construir sistemas que perduren en el tiempo.  

El \gls{ACSI} fue dise�ado 
en base al paradigma \gls{O-O-es}, con una estructura muy sencilla (desde el
punto de vista inform�tico) basado en un modelo jer�rquico de la informaci�n.
Su principal objetivo, hablando en t�rminos de dise�o de sofware,  
es desacoplar el dise�o del sistema (el proyecto) de 
la implementaci�n tecnol�gica (los protocolos), 
utilizando el patr�n de dise�o \gls{O-O-es} interface-implementaci�n.

La separaci�n de la interfaz (IEC 61850--7--2) de la implementaci�n 
(IEC 61850--8--x e IEC 61850--9--x) es el compromiso m�s importante que 
la norma IEC 61850 ha tomado para la construcci�n de sistemas a prueba de
futuro. Si bien una buena cantidad de clases del \emph{ACSI} 
son mapeadas directamente al protocolo MMS (osea, el ACSI en realidad no es tan
desacoplado del protocolo MMS), las interfaces existen, funcionan, 
y el proyectista del modelo IEC 61850 no tiene que preocuparse por como se
realiza el mapeado del ACSI a la pila de protocolos de la norma IEC 61850,
igualmente este aspecto es a prueba de futuro. En cuanto
a los protocolos, s�lo hay que
preocuparse por definir sus par�metros adecuadamente.  


El hecho de que a largo plazo se mantengan las interfaces \emph{ACSI} 
facilita una parte importante de la labor del proyectista, pues 
resulta beneficioso apostar por capacitarse en la norma IEC 61850, 
dado que la norma acompa�a de una manera m�s suave los cambios
tecnol�gicos.  

Resulta que el aporte de la norma IEC 61850 no es suficiente 
para construir sistemas a prueba de futuro, a�n resta el 
aporte del proyectista. 
El proyectista debe definir un modelo de datos consistente, 
aprovechando todo el modelo de datos sem�ntico 
proveida por la norma IEC 61850
(utilizando la menor cantidad de nodos l�gicos del grupo G siempre que sea
posible, por ejemplo)
y la posibilidad de describir 
en forma totalmente unificada, organizada y formal 
toda la estructura del sistema IEC 61850 (manteniendo 
un archivo SSD de todo el sistema).
Otros aspectos muy importantes que podr�a definir 
el proyectista para dise�ar sistemas a prueba de futuro
est�n mencionados en la secci�n
\ref{chEnfoque:ied-simplif-no-preconfig}




 


