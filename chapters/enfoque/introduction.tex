\section{Introducci�n}

La norma IEC 61850 propone, de manera flexible, 
una serie de pasos alternativos para abordar el proceso 
de construcci�n de los sistemas y redes de comunicaci�n 
en las subestaciones. Un mismo problema puede ser resuelto 
de diversas maneras, y esto puede ser evidenciado 
analizando el flujo de trabajo propuesto por las 
herramientas de ingenier�a disponibles actualmente 
en el mercado.



%Este cap�tulo describe como el autor 
%de este trabajo ha resuelto 
%el problema descripto en el 
%cap�tulo \ref{cap:planteamiento-del-problema}.

Tras la experiencia obtenida 
en el PTI durante la realizaci�n 
de este trabajo \cite{PTI:SESEP2010}, 
el autor ha seleccionado  
el proceso de ingenier�a que m�s se adapte 
a las necesidades encontradas para 
la elaboraci�n del modelo adecuado 
para los sistemas de comunicaci�n y redes 
del regualdor de velocidad de una unidad generadora 
t�pica de Itaipu.


El enfoque que se presenta en este cap�tulo 
trata de adecuarse a la realidad del entorno 
de mercado y de los recursos disponibles, 
con el objetivo de obtener un dise�o 
que en un futuro ayude a la construcci�n
del nuevo sistema IEC 61850 en las
unidades generadoras de Itaipu Binacional. 
Un enfoque adecuado permite obtener un dise�o 
que cubra las necesidades reales del sistema,
y haciendo realidad la promesa de la norma IEC 61850:
sistemas a prueba de futuro. 

El dise�o de las plantillas de tipos de datos 
y la definici�n de sus correspondientes instancias en los 
IEDs se puede realizar de diversas formas. El
enfoque presentado en este trabajo tiene como 
objetivo estar en armon�a con el enfoque 
utilizado en la Itaipu Binacional en  
la primera especificaci�n de equipos IEC 61850 
para la linea de 500KV \cite{Itaipu:linea500KV}, 
cuya licitaci�n internacional se ha hecho 
p�blica en octubre del 2010. Adem�s se 
propone enriquecer dicho enfoque con la 
especificaci�n de los Data Objects 
y sus respectivas instancias en IEDs, 
los cuales, seg�n la visi�n del autor de este trabajo, 
ayudar�a a mejorar el dise�o 
de sistemas a prueba de futuro. 

El modelo de informaci�n de la norma IEC 61850 
se clasifica como un sistema 
\gls{O-O-es}. Es necesario tener 
una clara comprensi�n de la tecnolog�a \gls{O-O-es} 
para aplicar correctamente el proceso de ingenier�a
propuesto en este cap�tulo. Por esta raz�n, 
esta secci�n describe en forma bien selectiva la
tecnolog�a \gls{O-O-es} en la cual 
la norma IEC 61850 
se ha apoyado. No se describen todos los principios 
orientados a objetos, s�lo se presentan 
los fundamentos necesarios para entender el modelo 
de informaci�n con el cual se contruy� la 
norma IEC 61850. Se provee una descripci�n detallada de 
los fundamentos del paradigma \gls{O-O-es}  seleccionados
a trav�s de implementaciones pr�cticas en lenguajes de 
programaci�n \gls{O-O-es} populares \cite{Java:specification},  
\glsentrytext{UML-es}  \cite{UML2:2009}
y se proveen materiales de referencia 
para su profundizaci�n.  















