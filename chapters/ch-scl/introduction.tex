\section{Introduction}

The \glspl{IED} in conformance with the 
IEC 61850 allow a communication related 
configuration with 
a standarized description language 
called \gls{SCL}, and they are specified 
in the part 6 of 
the IEC 61850 standard \cite{IEC61850-6:2004}.
The \gls{SCL} files, i.e.,\gls{SCD}, \gls{ICD} and \gls{SSD} 
files describes an instance of the 
\gls{SCL} classes in a serialized form 
and standarized description of constraints and object structure. 
\todo[color=green!40]{61850, parte6, cl6.1, \textparagraph 3}

This chapter describes the model defined 
in the IEC 61850-6 \cite{IEC61850-6:2004}. This 
point of view will be useful for the 
nexts chapters where the SAS objects will be 
modelled in conformance with this standard. 

The classes described in this chapter are part 
of the IEC 61850-6 standard. In the standard, 
theses classes are represented by \gls{XSD} files,
and \gls{UML}. This chapter aims to improve 
the understandability of the \gls{SCL} specification,   
%A good comprehension of the IEC 61850 Logical Nodes 
%and communication services models engineered 
%that result in a \gls{SCD} file 
to describes the steps of the \gls{SCL} engineering
and to analyze a real case describing their 
more important parts. 

%A good comprehension of t 

\todo{agregar mas cosas aca..}


\todo[inline]{en la breve introduccion del capitulo
debo explicar de que yo no mode\'e estas clases, 
sino que simplemente las estoy representando
de una manera conveniente para que sea entendible,
yo no invente nada aca, los graficos uml pertenecen
directamente al SCL.xsd definido en la norma,
yo simplemente represento los xsd en uml
para que sea mas facil explicar la estructura
y los constraints que exiran en el 
modelado de objetos basados en 
las clases uml descriptas en este capitulo}


