\section{Introduction}

The \glspl{IED} in conformance with the 
IEC 61850 allow a communication related 
configuration with 
a standarized description language 
called \gls{SCL}, and they are specified 
in the part 6 of 
the IEC 61850 standard \cite{IEC61850-6:2004}.
The \gls{SCL} files, i.e., 
\gls{SCD}, \gls{ICD}, \gls{SSD} and \gls{CID} 
files describes an instance of the 
\gls{SCL} classes in a serialized form 
and standarized description of constraints and object structure. 
\todo[color=green!40]{61850, parte6, cl6.1, \textparagraph 3}

This chapter describes the model defined 
in the IEC 61850-6 \cite{IEC61850-6:2004}. This 
point of view will be useful for the 
nexts chapters where the Itaipu Hydro Power Plant 
functions will be modelled in conformance 
with this standard. 

All the classes described in this chapter are part 
of the IEC 61850-6 standard, and the SAS 
engineering are bounded to the right instantiation 
of the classes described here. A comprehensive explanation 
about \gls{O-O} systems (classes, instances and more) 
are provided in the chapter \ref{ch:ch-oop}.
In the standard, 
the classes mentioned above are represented 
by \gls{XSD} files,
and simplified \glspl{UML}. This chapter aims to 
provide a alternative approach for 
the understandability of the \gls{SCL} specification, 
%A good comprehension of the IEC 61850 Logical Nodes 
%and communication services models engineered 
%that result in a \gls{SCD} file 
\todo{aqui: revisar la gramatica}
describe the steps of the \gls{SCL} engineering 
and analyze a real case SCL by describing their 
more important parts, and providing the 
complete \gls{UML} class diagram of the IEC 61850-6.



