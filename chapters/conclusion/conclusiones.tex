\section{Conclusiones}

Este Trabajo Final de Grado comienza 
en el cap�tulo \ref{ch:introduction} 
con una breve introducci�n a la norma IEC 61850. Esta 
introducci�n describe
la importancia de la norma, 
las nociones b�sicas de la 
norma, 
su aplicaci�n en las hidroel�ctricas, 
dando mayor enfoque a la explicaci�n del 
modelo de informaci�n y servicios que ofrece 
la IEC 61850, y a la explicaci�n del 
lenguaje de configuraci�n de estos 
sistemas de comunicaci�n y redes.

Un objetivo importante de este trabajo 
es interpretar la norma IEC 61850, 
identificando sus puntos fuertes y d�biles
para su aplicaci�n en centrales hidroel�ctricas. 
Acorde a esto, en los cap�tulos 
\ref{cap:planteamiento-del-problema} y
\ref{ch:enfoque-del-autor}
se identificaron los problemas y los desaf�os 
existentes durante el proceso de ingenier�a IEC 61850
en sistemas de automatizaci�n de centrales hidroel�ctricas,
dando un primer paso en la armonizaci�n de 
la IEC 61850 con la IEEE 830 \cite{IEEE:830-1998}, 
y visualizando 
las ventajas que tendr�a la adaptaci�n y
aplicaci�n de esta norma IEEE
para la identificaci�n de los requisitos 
que deben cumplir los sistemas IEC 61850
a ser dise�ados en una forma totalmente sistem�tica,
y se explica al lector el enfoque adoptado por el 
autor de este trabajo durante
todo el proceso de ingenier�a correspondiente
a los objetivos del trabajo. La investigaci�n
descripta en el cap�tulo 
\ref{ch:enfoque-del-autor}
ha mejorado la comprensi�n de la norma 
IEC 61850, en especial del proceso de 
ingenier�a produce resultados en 
el formato formal 
definido en IEC 61850--6 \cite{IEC61850-6:2004}.
Un an�lisis del uso de ICDs para 
especificaci�n de equipamientos IEC 61850 
ha sido profundizado, ofreciendo 
datos y ejemplos concretos, describiendo 
sus ventajas y desventajas.

De esta manera, en el cap�tulo \ref{cap:model}
se aborda el objetivo principal de este trabajo: 
El dise�o del modelo IEC 61850 del regulador de velocidad 
de una unidad generadora t�pica de la 
Central Hidroel�ctrica Itaipu. Este  
dise�o se realiza aplicando el enfoque
propuesto en el proceso de ingenier�a  
explicado en el cap�tulo 
\ref{ch:enfoque-del-autor}.
En este cap�tulo se propone la inclusi�n
de dos nodos l�gicos a la norma IEC 61850,
de modo a enriquecer e ir cubriendo todas 
las necesidades del \gls{SAS}, 
y en las secciones \ref{sec:LNs-del-IEDRV}
en adelante se gener� en forma autom�tica
acorde a las directrices de la norma IEC 61346,
descriptas en la secci�n \ref{sec:doc-automatica}, 
tomando como informaci�n estructurada el 
resultado del dise�o, que se present� 
en formato SCL en el 
anexo \ref{app:resultados2-codigos-SCL}.
