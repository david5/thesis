\section{Organizaci�n de este documento}

El formato y la estructura de este documento 
fueron elaborados siguiendo las directrices 
dadas por la norma IEC 61802 -- \emph{Preparation 
of documents used in electrotechnology} \cite{IEC61082-1:2004} y 
el layout escrito en \LaTeXe,  de \emph{Athena Online -- 
Massachusetts Institute of Technology} \cite{MIT-athena:2010}.  


Este documento est� organizado 
de la siguiente forma:

El cap�tulo \ref{ch:informaciones-generales} 
provee una introducci�n sobre este trabajo, 
los objetivos de este trabajo, la justificativa, 
y otras informaciones relacionadas al 
proceso de investigaci�n.

El cap�tulo \ref{ch:introduction}
realiza una breve introducci�n general 
a la norma IEC 61850, 
resaltando su uso en las Centrales 
Hidroel�ctricas, de modo a que 
el lector sin experiencia en la norma IEC 61850
pueda seguir leyendo los cap�tulos posteriores. Tambi�n 
presenta los aspectos t�cnicos de un regulador de velocidad 
utilizado actualmente en Itaipu.

El planteamiento del problema a ser resuelto en 
este trabajo, como as� tambi�n la descripci�n 
de los desaf�os encontrados son presentados 
en el 
cap�tulo \ref{cap:planteamiento-del-problema}.
En este cap�tulo, respondiendo directamente
a la identificaci�n de los problemas, 
se comenta 
la importancia de resolver estos 
problemas de dise�o encontrados 
al aplicar la norma IEC 61850 en 
centrales hidroel�ctricas.

En el cap�tulo \ref{ch:enfoque-del-autor}
se describe como el autor 
ha enfocado el proceso de dise�o 
del modelo IEC 61850 (descripto en las 
secci�n \ref{sec:objetivos-investigacion})
y como ha resuelto los problemas planteados 
en el cap�tulo anterior.

En el cap�tulo \ref{cap:model}
se presentan los resultados: 
el dise�o del modelo IEC 61850 
del regulador de velocidad 
de una unidad generadora t�pica de la 
Central Hidroel�ctrica Itaipu. 

Finalmente, en el cap�tulo \ref{ch:conclusiones}
se discuten las conclusiones 
y el futuro alcance de esta 
investigaci�n. 


