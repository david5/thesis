\section{T�cnicas y metodolog�a de investigaci�n}

	\subsection{Investigaci�n bibliogr�fica}
	La investigaci�n bibliogr�fica consisti�
	en el estudio de la norma IEC 61850, en especial las
	definiciones de modelos de informaci�n definidos en la parte IEC 61850--7--4--10
  	\emph{``Basic communication structure for substation and feeder equipment -- 
  	Compatible logical node classes and data classes -- Hydro Power Plants''}, 
  	de los servicios de intercambio de informaci�n para diferentes funciones 
	definidos en el apartado IEC
  	61850--7--2 \emph{``Basic communication structure for substation and feeder
  	equipment -- Abstract communication service 
  	interface (ACSI)''} \cite{IEC61850-7-2:2003}, la
  	implementaci�n de dichos servicios de comunicaci�n a trav�s de protocolos
  	especificados en la parte IEC 61850--8--1 \emph{``Specific Communication
  	Service Mapping (SCSM) Mappings to MMS (ISO 9506--1 and ISO 9506--2) and to
  	ISO/IEC 8802-- 3''} \cite{IEC61850-8-1:2004} y 
  	la configuraci�n y descripci�n
  	formal de todas ellas a trav�s del apartado IEC 61850--6 \emph{``Basic communication structure
  	for substation and feeder equipment -- Configuration description language
  	for communication in electrical 
  	substations related to IEDs''} \cite{IEC61850-6:2004} 	
  	para las necesidades de la unidad generadora, y otros
  	documentos, normas, y tecnolog�as relacionadas.
  	Tambi�n se ha accedido la base de datos de ACM a trav�s 
  	del Portal de la Capes.
  	
	\subsection{Investigaci�n de campo}
		Incluy� el levantamiento de informaciones del
		generador de Itaipu, la identificaci�n de las funciones de automatizaci�n
		inherentes al regulador de velocidad de la unidad generadora y otras
		documentaciones y estad�sticas relacionadas al caso en estudio. 
% 		Tambi�n
% 		incluye la simulaci�n en una red IEC 61850 utilizando las herramientas de
% 		ingenier�a disponibles en el mercado para tal efecto.
% 		\todo[inline]{eso de la simulaci�n en una red est� por verse. Es probable de
% 		que quite eso de ac�} - YA QUIT�
	\subsection{An�lisis de herramientas de ingenier�a}
		Incluy� el an�lisis de herramientas de 
		ingenier\'ia especializadas en la norma IEC
		61850 disponibles en el mercado.

	\subsection{Propuesta de extensi\'on/complementaci\'on de los nodos l\'ogicos}
		%e ICDs} 
		Preparaci�n de una propuesta de extensi�n de los nodos l�gicos
		para cubrir las necesidades de la Itaipu Binacional.
		
% 		Por ejemplo, ZAXN, insuficiente para la alimentaci\'on AC y DC de los
% 		servicios auxiliares de Itaipu. 
% 		\todo[inline, color=blue!50!red!50]{este nodo logico 
% 		es solo un ejemplo, 
% 		la idea es que, si se encuentran funcionalidades no contempladas  
% 		por la norma, pero son necesarias, sea propuesta en los tissues.
% 		El ZAXN es solo un ejemplo, pero en realidad voy a cambiar este 
% 		ejemplo, o dejo asi, no se todavia, dado que no voy a modelar 
% 		nada de los servicios auxiliares.		
% 		esto debo revisarlo mas adelante, no tiene 
% 		prioridad inmediata y es un punto que dejare pendiente
% 		para esta primera mitad del tiempo de mi investigacion}

	%\subsection{Metodolog\'ia de aplicaci\'on }
	\subsection{Enfoque de la ingenier�a IEC 61850}
		Se estructur� una propuesta de metodolog\'ia de aplicaci\'on
		de la norma IEC 61850 en la automatizaci\'on de hidroel\'ectricas
		a trav�s de la combinaci�n de 
		herramientas de ingenier\'ia (disponibles en el mercado) que mejor se
		adapten a la automatizaci\'on de unidades generadoras en conformidad con 
		esta norma.

