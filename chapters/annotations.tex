\chapter*{Annotations}

Auxiliary chapter (not is part of this document content).
En esta parte explico sobre la plataforma latex que utilizo, sobre la
sistematizaci\'on que utilizo para redactar este documento, por ejemplo, que
significan los comentarios que hago con el comando TODO, etc.

\section{Convenci\'on de colores para los comentarios hechos en este documento}
	Un comentario de color 
	\emph{NARANJADO}
	\todo{Ejemplo}   
	indica las cosas que me faltan hacer.\\

	Un comentario de color 
	\emph{VERDE CLARO}
	\todo[color=green!40]{green!40}
	indica en que parte de la norma IEC 61850 se est\'a utilizando la teor\'ia
	comentada. Indica la relaci\'on del contenido bibliogr\'afico de conceptos con
	la norma. 
	Un ejemplo de uso de este tipo de comentarios en mi trabajo:
	En redes, cuando hablo de un concepto, por ejemplo, \emph{network
	reliability}, eso se habla en la norma, en la parte IEC 61850-xx, entonces yo
	indico en el comentario de que ese concepto se menciona en la tesis. 
	Esto es \'util para mi pues me sirve para justificar por que inclu\'i la 
	descripci\'on de alg\'un concepto en el trabajo, principalmente en el marco
	te\'orico, que utilizar\'e para describir todos los conocimientos necesarios
	para abordar la norma (por ejemplo: Redes, Programaci\'on orientada a
	objetos, UML).\\
	
	Los colores en \emph{LILA}
	\todo[inline, color=blue!50!red!50]{color=blue!50!red!50}
	indican peque\~nas variaciones de palabras que podr\'ia hacer pero no son de
	alta prioridad ni obligatorias de corregir (a mi punto de vista), pero
	la modificaci\'on de las partes comentadas mejorar\'ia la redacci\'on del
	texto.
	

\section{Im\'agenes por agregar al trabajo}
	Por convenci\'on, utilizar\'e la siguiente imagen
	\missingfigure[color=green!40]{prueba del comando de falta de figura}
	para indicar de que est\'a faltando incluir la imagen en este lugar. Durante
	el proceso de redacci\'on se ir\'a agregando el contenido faltante




%%Agrega la lista de cosas (comentarios) que hay que hacer
\listoftodos

