\section{Desaf�os encontrados al realizar el dise�o}

Debido a que durante el transcurrir de este trabajo no se ha encontrado en el
mercado ning�n IED regulador de velocidad de turbinas hidroel�ctricas que
implemente la norma IEC 61850-7-4-10 \cite{IEC61850-7-410:2007} se ha vuelto 
dif�cil hallar un punto de partida para 
definir los procesos de ingenier�a adecuados para el efecto. De
hecho, las herramientas de ingenier�a tampoco proporcionan un soporte completo
para los proyectos IEC 61850 en hidroel�ctricas.      

Otro problema importante fue la falta literatura sobre la norma IEC
61850--7--4--10 \cite{IEC61850-7-410:2007}. A la fecha de culminaci�n de este
trabajo no se encontraban disponibles documentos p�blicos oficiales publicados por la IEC que describan
la combinaci�n adecuada de los nodos l�gicos definidos por la norma IEC
61850--7--4--10 \cite{IEC61850-7-410:2007} para conformar las funciones de
monitoreo y control correspondientes. En el 
�mbito de subestaciones las cl�usulas 8, 9 y 11 de la
IEC 61850--5 \cite{IEC61850-5:2003} proveen un punto de partida muy importante
para el modelado de las funciones del 
sistema gracias a la combinaci�n correcta de los nodos l�gicos
definidos en la IEC 61850--7--4 \cite{IEC61850-7-4:2003}. El autor de este
trabajo tiene entendido que los documentos oficiales que describen aspectos relevantes 
del  dise�o de modelos de sistemas IEC 61850 para hidro�ctricas est�n en proceso
de elaboraci�n, y ser�n publicados bajo la serie IEC 61850--7--510 \cite{IEC61850-7-510:2009}.
